\documentclass{beamer}

\mode<presentation> {
  \usetheme{CambridgeUS}
  \setbeamercovered{transparent}
}

\usepackage{ucs}
\usepackage[utf8x]{inputenc}
\usepackage[english]{babel}
\usepackage{palatino}
\usepackage{graphicx}
\usepackage{hyperref}

\usepackage{minted}

\title{Qt Networking}
\author{Miroslav Jaroš}
\institute[Pb173]{Pb173 Crypto}
\date{11.~4.~2016}

\begin{document}
    \begin{frame}
        \titlepage
    \end{frame}
    \begin{frame}
        \frametitle{Agenda}
        \tableofcontents
    \end{frame}

    \section{Networks in general}
    \begin{frame}
        \frametitle{Computer networks}
        \begin{itemize}
            \item Connects computers for information exchange
            \item Computers communicates using specified protocol
            \begin{itemize}
                \item Protocol uses each other
                \item Http works over TCP
            \end{itemize}
            \item \texttt{ISO/OSI} model defines 7 layers
            \item \texttt{TCP/IP} -- Internet
        \end{itemize}
    \end{frame}
    \begin{frame}[allowframebreaks]
        \small
        \frametitle{TCP/IP}
        \begin{itemize}
            \item \texttt{IP} Internet protocol
            \begin{itemize}
                \item Addressing computers over network
                \item Works on 3rd layer in ISO/OSI
                \item \texttt{IPv4} 32 bit address commonly written in 4 digital numbers \texttt{127.0.0.1}
                \item \texttt{IPv6} 128 bit address commonly written as 8 hexadecimal numbers \texttt{::1}
                \item Also adds many other features for peer to peer communication
            \end{itemize}
            \framebreak
            \item \texttt{TCP Transmission control protocol}
            \begin{itemize}
                \item Reliable and ordered delivery
                \item Works on 4th layer of ISO/OSI
                \item Every connection is quadruple \texttt{\(SOURCE\_HOST, SOURCE\_PORT, TARGET\_HOST, TARGET\_PORT\)}
                \item Data are delivered complete and in same order as they were transmitted
                \item You can't use TCP for broadcasting
            \end{itemize}
            \framebreak
            \item \texttt{UDP User datagram protocol}
            \begin{itemize}
                \item Nonreliable delivery
                \item Connection is identified as in TCP but there is no control of validity
                \item Best-effort protocol, there is no retransmission or ordering of data
                \item On the other hand UPD is in many cases much faster than TCP due to almost nonexisting messaging overhead
                \item Can be used for broadcasting, multicasting, anycasting, and unicasting
            \end{itemize}
        \end{itemize}

    \end{frame}
    \section{Networks in Qt}
    \begin{frame}{Networks in programming}
        \begin{itemize}
            \item Networking is service provided by OS
            \item Any network acitons are accessible via OS API (Win 32, POSIX)
            \item Therefore network code is strongly depends on target platform
            \item Priviledged ports
            \item Nondeterminism: you can't tell when or on which route will be your message delivered
            \item Paralelism: Many of network events can happen simultaneously
        \end{itemize}
    \end{frame}
    \subsection{Basic concept}
    \begin{frame}[allowframebreaks]{Networks and Qt}
        \begin{itemize}
            \item Creates multiplatform network interface and tools
            \item All netwokr classes and functions are contained in module network
            \item Just add \texttt{Qt += network} in your .pro file
            \item It contains classes to handle Tcp and Udp connections, SSL connections and
            even websockets
            \framebreak
            \item Most important classes are:
            \begin{itemize}
                \item \texttt{QTcpSocket} handles single Tcp connection
                \item \texttt{QTcpServer} creates listening socket and handles new icnomming connections
                \item \texttt{QUdpSocket} handles udp connections -- there's no need for UdpServer
                \item \texttt{QSslSocket} handles encrypted connections
            \end{itemize}
        \end{itemize}
    \end{frame}
    \subsection{Signals and Slots}
    \begin{frame}{QtNetwork and Signals}
        \begin{itemize}
            \item You should be probably aware of Signals and Slots concept
            \item QtNetwork uses signals and slots in same manner as the rest of library
            \item Sockets emits signals that induces the change of state or any event that happens on socket
            \item The most important for you are
            \begin{itemize}
                \item \texttt{readyRead} Data were delivered to socket and are ready to be read
                \item \texttt{disconnected} Socket host has disconnected
            \end{itemize}
        \end{itemize}

    \end{frame}
    \subsection{Demo}
    \begin{frame}{Demo}
        \begin{itemize}
            \item Simple Client-Server architecture
            \item Can be found here \url{https://github.com/mijaros/QtNetworks.git}
        \end{itemize}
    \end{frame}
    \begin{frame}{Links}
        \begin{itemize}
            \item \url{http://doc.qt.io} Qt documentation
            \item \url{http://doc.qt.io/qt-4.8/signalsandslots.html} Introduction to signals and slots principle
            \item \url{http://doc.qt.io/qt-5/qtnetwork-index.html} Network~module documentation
        \end{itemize}
    \end{frame}
    \begin{frame}{The end}
        \Large Thank you for attention
    \end{frame}

\end{document}
